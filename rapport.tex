%%%%%%%%%%%%%%%%%%%%%%%%%%%%%%%%%%%%%%%%%%%%%%%%%%%%%%%%%%%%%
%% Based on a TeXnicCenter-Template, which was             %%
%% created by Christoph Börensen                           %%
%% and slightly modified by Tino Weinkauf.                 %%
%%                                                         %%
%% Then, a third guy - me - put in some modifications.     %%
%%                                                         %%
%% IFT2035 - Rapport TP1                                   %%
%%%%%%%%%%%%%%%%%%%%%%%%%%%%%%%%%%%%%%%%%%%%%%%%%%%%%%%%%%%%%

\documentclass[letterpaper,12pt]{scrartcl}
% Optimised for letter. Add ",twosides" to use the two-sides layout.

% Margins
    \usepackage{vmargin}
    \setpapersize{USletter}
    \setmargins{2.0cm}%	 % Left edge
               {1.5cm}%  % Top edge
               {17.7cm}% % Text width
               {21.0cm}% % Text height
               {14pt}%	 % Header height
               {1cm}%    % Header distance
               {0pt}%	 % Footer height
               {2cm}%    % Footer distance
				
% Graphical bugfix (about footnotes)
    \usepackage[bottom]{footmisc}

% Fonts and locale
	\usepackage{t1enc}
	\usepackage[utf8]{inputenc}
	\usepackage{times}
	\usepackage[francais]{babel}
	\usepackage{SIunits}
	\usepackage{amsmath}

	\AtBeginDocument {%
	    \renewcommand\tablename{\textsc{Tableau}}
	}

% Graphics
	\usepackage[pdftex]{graphicx}
	\usepackage{color}
	\usepackage{eso-pic}
	\usepackage{everyshi}
	\renewcommand{\floatpagefraction}{0.7}

% Enable hyperlinks
	\usepackage[pdfborder=000,pdftex=true]{hyperref}
	
% Table layout
	\usepackage{booktabs}

% Caption
	\usepackage{ccaption}
	\captionnamefont{\bf\footnotesize\sffamily}
	\captiontitlefont{\footnotesize\sffamily}
	\setlength{\abovecaptionskip}{0mm}

% Header and footer settings
	\usepackage{scrpage2} 
	\renewcommand{\headfont}{\footnotesize\sffamily}
	\renewcommand{\pnumfont}{\footnotesize\sffamily}

% Pagestyles
	\defpagestyle{cb}{
		(\textwidth,0pt) % Sets the border line above the header
		{\pagemark\hfill\headmark\hfill} % Doublesided, left page
		{\hfill\headmark\hfill\pagemark} % Doublesided, right page
		{\hfill\headmark\hfill\pagemark} % Onesided
		(\textwidth,1pt)} % Sets the border line below the header
		{(\textwidth,1pt) % Sets the border line above the footer
		{{\it Rapport TP2 (IFT2035)}\hfill Sulliman Aïad et François Poitras} % Doublesided, left page
		{Sulliman Aïad et François Poitras\hfill{\it Rapport TP2 (IFT2035)}} % Doublesided, right page
		{Sulliman Aïad et François Poitras\hfill{\it Rapport TP2 (IFT2035)}} % One sided printing
		(\textwidth,0pt) % Sets the border line below the footer
	}

% Empty pages style
	\renewpagestyle{plain}
		{(\textwidth,0pt)
			{\hfill}{\hfill}{\hfill}
		(\textwidth,0pt)}
		{(\textwidth,0pt)
			{\hfill}{\hfill}{\hfill}
		(\textwidth,0pt)}

% Footnotes
	\renewcommand{\footnoterule}{\rule{5cm}{0.2mm} \vspace{0.3cm}}
	\deffootnote[1em]{1em}{1em}{\textsuperscript{\normalfont\thefootnotemark}}

\pagestyle{plain}

\begin{document}
	\begin{center}
		\vspace{2cm}

		{\Huge\bf\sf Rapport du Travail Pratique 2}

		\vspace{0.5cm}

		{\bf\sf (TP2)}

		\vspace{4cm}

		{\bf\sf Par}

		\vspace{0.5cm}{\large\bf\sf Sulliman Aïad et François Poitras}

		\vspace{2cm}

		{\bf\sf Rapport présenté à}

		\vspace{0.5cm}{\large\bf\sf M. Marc  Feeley}

		\vspace{2cm}

		{\bf\sf Dans le cadre du cours de}

		\vspace{0.5cm}{\large\bf\sf Concepts des langages de programmation (IFT2035)}

		\vspace{\fill}
		Remis le \today

		\vspace{0.5cm}Université de Montréal
	\end{center}
	
	\newpage

	\pagestyle{cb}
	
	\tableofcontents

	\newpage
	
	\section{Fonctionnement du programme}
	Tout comme dans le TP1, le programme est une calculatrice à précision infinie en notation postfixe. Le programme débute dans un mode d'attente, avec le signe «?», pour différencier des instructions de l'interpréteur Gambit. L'utilisateur peut entrer une expression postfixe, demander la valeur d'une variable, ou encore affecter le résultat d'une expression à une variable. Notons qu'il est possible d'utiliser des variables dans les expressions et notons que les variables peuvent seulement être des lettres minuscules.

		En cas de problème, soit avec la syntaxe ou avec l'utilisation de variables non-initialisées, l'utilisateur est informé de l'erreur qu'il a commise et le programme attends la prochaine instruction. Si des variables étaient utilisées avec une mauvaise syntaxe, leur valeur ne sera pas initialisée ou modifiée.
	
	%% END OF {Fonctionnement du programme} %%
	
    
	\section{Problèmes de programmation}
		\subsection{Traitement d'une expression de longueur quelconque }
        Le traitement d'une expression commence par l'appel à la fonction \textit{process} avec une liste vide comme argument. Cette liste représente la pile qui va servir au calul. Notons aussi que l'autre argument de la fonction est l'expression elle-même, traitée par la fonction \textit{split}. Cette dernière fonction a pour effet de transformer une chaîne de caractères en une liste. Pour ce faire, elle utilise une récursion en forme itérative.
        
        La fonction \textit{split} utilise un prédicat pour évaluer si le caractère courant est à utiliser comme séparation ou non. Dans le cas qui nous intéresse, le caractère de séparation est un espace, ce qui signifie que le prédicat est la fonction pré-définie \textit{char-whitespace?} Une fonction lambda interne est appelée récursivement avec une liste vide, la liste à construire et une autre liste vide. Les deux listes vides représentent respectivement le résultat total et le mot en cours de traitement. 
        
        Si il ne reste plus rien à traiter et que le mot en cours est également vide, cela signifie que tout a été traité et on peut donc retourner la liste complète. Si la liste en cours n'est pas vide, on \textit{cons} l'inverse de la liste en cours à l'inverse de la liste complète. Cela est nécéssaire car les caractères sont ajoutés au début de la liste complète. Pour avoir un résultat qui n'est pas inversé, on doit inverser la liste complète. Un raisonnement similaire explique l'inversion de la liste en cours.
        
        D'un autre coté, si il restait des éléments à traiter, on regarde si le caractère courant valide le prédicat. Si oui, on fait un appel récursif de la fonction lambda, en lui passant, dans l'ordre, la liste complète, la concaténation de l'inverse et de la liste en cours et le \textit{cdr} de ce qui reste à traiter. Si le prédicat n'est pas vérifié, l'appel récursif est fait avec la liste complète, le \textit{cdr} de ce qui reste et la concaténation du caractère courant à la liste en cours.
        
        Toutes ces appels récursifis font en sorte qu'une fois l'expression entièrement parcourue, la liste complète contient tous les mots de l'expression, selon le prédicat passé en argument. On peut alors passer au calcul de l'expression.
        \subsection{Calcul de l'expression}
        Le calcul se fait de manière récursive, Scheme oblige. La fonction \textit{process} vérifie tout d'abord s'il reste des éléments à traiter. Si il n'y en a pas, on vérifie le nombre d'éléments sur la pile. S'il est supérieur à 1, cela signifie qu'il y a une erreur syntaxique dans l'expression. Si il y a un seul élément, celui si est le résultat du calcul et il est affiché. Si aucun élément n'est présent, cela signifie qu'aucun calcul n'a été demandé.
        
        Si il reste des éléments à traiter, on regarde la nature de l'élément courant. Si c'est un nombre, on l'enregistre sur la pile et on passe au traitement du prochain élément. S'il s'agit d'un opérateur, on s'assure qu'il y a suffisament d'éléments sur la pile avant de dépiler deux fois et d'empiler le résultat du calcul.
        \subsection{Affectation des variables}
        Toujours dans la fonction \textit{process}, nous arrivons à traiter les variables. Si le mot courant est le signe '=' suivi d'une lettre, on affecte la valeur contenue sur la pile à cette variable et on appelle récursivement le prochain mot. Il est aussi possible qu'une variable soit dans le calcul. Si c'est le cas, on utilise \textit{assoc} pour récuprerer sa valeur dans le dictionnaire défini au début de l'exécution du programme. Si la variable demandée n'a pas de valeur, on affiche un message d'erreur.
        \subsection{Affichage des résultats et traitement des erreurs}
        Si tout c'est bien déroulé, on affiche le résultat du calcul ou la valeur de la variable demandée et le programme attends la prochaine expression. Si une variable a été affectée suite à un calul, on affiche aussi le résultat dudit calcul. Dans le cas où aucune des conditions plus haut n'a été vérifiée, cela signifie qu'une erreur s'est glissée dans l'expression. Un message informe l'usager de cette erreur et le programme se met en attente d'une nouvelle expression. Il en va de même pour l'appel à une variable non-initialisée.
	%% END OF {Problèmes de programmation} %%
	\section{Comparaison entre C et Scheme}
	
	L'utilisation d'un paradigme strictement fonctionnel a posé quelques difficultés au niveau du traitement des expressions, mais a été très bénéfique dans la gestion globale des ressources. Notre code Scheme est également beaucoup plus court. À titre de comparaison, notre code C comporte environ 1000 lignes de code, contre 150 pour Scheme. Cette différence s'explique surtout par le charactère intrinsèquement récursif de Scheme et par l'absence de gestion manuelle de pointeurs.
	
	Dans le language fonctionnel, il a été beaucoup plus facile d'implémenter des nombres de longueur arbitraire, car cette fonctionalité est déjà incluse dans Scheme. En C, il nous a fallu créer une structure pour gérer des nombres potentiellement très grands, avec tout les problèmes qui viennent avec. Ces problèmes incluent la gestion manuelle de la mémoire et l'utilisation de pointeurs. Par contre, étant donné la philosophie minimaliste de Scheme, il nous a fallu implémenter nous même des fonctions de base présentes dans les extensions autorisées dans le TP1. La fonction \textit{split} est un bon exemple d'une telle fonction. En C, nous avions \textit{strtok} pour nous aider. Dans un language impératif, implémenter cette fonction aurait été plus simple, car un traitement non-récursif est plus simple à implémenter. 
	
	De l'autre coté, la difficulté en C a été de gèrer manuellement des pointeurs et la mémoire sur des structures que nous avions nous même créées. Il a fallu être extrêmement vigilants sur l'utilisation des \textit{malloc} et des \textit{free}. En particulier, nous avons prété attention à libérer la mémoire au bon moment, que ce soit dans la gestion de la pile ou dans la gestion de très grands nombres.
	
	Dans les deux languages, il nous a fallu faire une structure de pile. En Scheme, il s'agit d'une simple liste que nous utilisons récursivement, à l'aide des fonctions \textit{car,crd,caar,cadr,etc.}. Il est relativement simple de dépiler en appelant la suite de la liste. Pour empiler, on peut utiliser \textit{cons}.  En C, la pile est une structure comportant une taille, une taille maximale et des données. Pour empiler, il faut utiliser un calcul de pointeurs en crééant un élément au premier index disponible dans les données. Pour dépiler, le processus inverse est utilisé, mais au lieu d'allouer, on libère de la mémoire.
	
	La dernière principale différence concerne la lecture de l'entrée. En C, on lit caractère par caractère et on alloue dynamiquement la mémoire au besoin. Ensuite, on peut commencer à traiter l'expression en utilisant \textit{strtok}. Les résultats peuvent être affichés d'un coup avec des \textit{printf}. En Scheme, chaque caractère est affichée à l'aide de la boucle \textit{for-each}.
\end{document}